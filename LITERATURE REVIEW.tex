\section{Literature Review}
    \begin{frame}{Literature Review}
In the first place, (Castillo, 2001) in his work entitled \textit{“Remittances: a cointegration analysis for the case of Mexico”} explains the elaboration of a:\par
\textit{Cointegration analysis that establishes a long-term relationship between remittances and a series of exogenous variables, which include the GDP of the United States, the GDP of Mexico, and the real exchange rate. A simulation exercise is presented through which the impact that changes in the explanatory variables would have on remittances is evaluated.}\par
\end{frame}
\begin{frame}{Literature Review}
Secondly, (Gómez and Ramírez, 2014) in the work titled: \textit{“Economic dynamics of remittances sent from Spain and the United States to Colombia between 2005-2013: a cointegration analysis”} analyze the following:\par
\textit{The purpose of this research is to analyze the behavior of remittances sent from Spain and the United States to Colombia in the last nine years, and specifically look for their long-term determinants. To achieve this, a validated cointegration analysis is implemented with an error correction method.}\par
\end{frame}

\begin{frame}{Literature Review}
In third place, (Islas and Moreno, 2011) in his research titled: \textit{“Determinants of the flow of remittances in Mexico, an empirical analysis”} where they analyze the following:\par
\textit{The macroeconomic variables that determine the flow of family remittances from the United States to Mexico. A hypothesis based on a broad revision of the theory on the flow of remittances is presented and an econometric exercise is carried out using Autoregressive Vectors with Error Correction\footnote{VARCE.}  to identify, from a synthetic perspective, the long-term determinants of family remittances that arrive in Mexico. The empirical analysis presents evidence that remittances are the consequence of an investment decision rather than altruism on the part of migrants. In addition, it was found that there is an inverse and significant relationship between the number of migrants and remittances, which could be an indicator that the stay of migrants in the United States is very long-term or permanent.}\par
	\end{frame}
\begin{frame}{Literature Review}
In fourth place, (Arias and Muñoz, 2019) in his research work entitled: \textit{“Main macroeconomic determinants of imports in the CAN\footnote{Comunidad Andina de Naciones.}   countries”} analyzes the following:\par
\textit{The purpose of this work is to identify the main macroeconomic determinants of imports given the negative trend that has occurred in the behavior of net exports in three of the four CAN countries, in addition to the relevance of imports within the national accounts of each country. The foregoing taking into account that the CAN is a subregional organization whose main purpose is to improve the standard of living and balanced development of the member countries through integration.}\par
	\end{frame}
\begin{frame}{Literature Review}
In fifth place, (Cruz and Salazar, 2013) in his work titled: \textit{“Remittances and economic growth: evidence for the Mexican economy”} studies the following:\par
\textit{The growth rate of remittances that arrive in Mexico since the mid-1980s has been higher than that of exports and that of other external capital flows such as foreign and portfolio investment. Therefore, it is relevant to identify whether remittances have contributed to the growth of the Mexican economy. To answer this question, we estimate a consumption function that includes remittances.}\par
In this analysis, four variables were used: private consumption (since, if remittances positively affect consumption, then this will also have a positive effect on economic growth), private investment\footnote{As proxy variable of income.} , the real exchange rate (given the relevance that has to modify the purchasing power of workers when it has significant variations) and remittances. 
\end{frame}