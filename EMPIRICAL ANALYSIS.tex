\section{Empirical analysis}
    \begin{frame}{Empirical analysis}
In previous sections, the series were shown in levels and individually. The variables in their logarithmic transformations are shown below:\par
As observed in the previous series, their seasonality is highlighted and with the help of dichotomous variables, the impact of this problem is reduced. The seasonally adjusted  
series are shown below:\par
In these series, the Unit Root test was applied using the Dickey-Fuller method, in which the hypotheses are as follows:\par
\begin{center}
\textbf{H0} = Has unit root.
\textbf{HA} = The series is stationary.
\end{center}
In the first place, the test was applied to Remittances, the results obtained \textbf{(-1.446)} refer to the fact that the test adding the trend is not rejected H0 \textbf{(-4.051 at 1 percent, -3.455 at 5 percent and -3.153 at 10 percent)} and the form of change in levels \textbf{(-1.826)} is not rejected H0 \textbf{(-2.367 at 1 percent, -1.661 at 5 percent and -1.291 at 10 percent)}. \par
Second, the test was applied to Mexico's GDP where the results \textbf{(-3.656 in trend and -1,349 in change in levels)} do not reject H0 in its 1 percent trend form \textbf{(-4.051 at 1 percent, -3.455 at 5 percent and -3.153 at 10 percent)} and its change in levels at 5 percent \textbf{(-2.367 at 1percent, -1.661 at 5 percent and -1.291 at 10 percent)}.\par
Third, the test was applied to the United States GDP where the results \textbf{(-2.914 in trend and -1.164 in change in levels)} do not reject H0 in its trending form \textbf{(-4.051 at 1 percent, -3.455 at 5 percent and -3.153 at 10 percent)} and in its change in levels \textbf{(-2.367 at 1 percent, -1.661 at 5 percent and -1.291 at 10 percent)}.\par 
Finally, the test was performed at the Real Exchange Rate where the results \textbf{(-3.492 in trend and -2.701 in change in levels)} do not reject H0 in its form with a trend of 1 percent \textbf{(-4.051 at 1 percent, -3.455 at 5 percent and -3.153 at 10 percent)} and in its change in levels if it rejects H0 \textbf{(-2.367 at 1 percent, -1.661 at 5 percent and -1.291 at 10 percent)}.\par
Since the series has structural changes, the Zivot and Andrews test is carried out considering the same hypotheses as the previous test. For Remittances in the first instance, the results for the form with change in the intercept and trend \textbf{(-4.783)} are not rejected H0 \textbf{(-5.57 at 1 percent, -5.08 at 5 percent and -4.82 at 10 percent)} and in the same way , in the way of minimizing lags \textbf{(-3.208)}, H0 is not rejected \textbf{(-5.34 at 1 percent, -4.80 at 5 percent and -4.58 at 10 percent)}.\par
Second, the test was performed on the Mexican GDP series where the results \textbf{(-4.250)} show that the test does not reject H0 with the change in the intercept and trend \textbf{(-5.57 at 1 percent, -5.08 at 5 percent and -4.82 at 10 percent)}, in the same way \textbf{(-4.689)} H0 at 5 percent is not rejected in its way of minimizing lags \textbf{(-5.34 at 1 percent, -4.80 at 5 percent and -4.58 at 10 percent)}.\par
Third, the test was performed on the US GDP series where the results \textbf{(-4.856)} show that the test does not reject H0 with the change in the intercept and trend to 5 percent \textbf{(-5.57 to 1 percent, -5.08 at 5 percent and -4.82 at 10 percent)}, but if it rejects \textbf{(-6.059)} H0 in its way of minimizing lags \textbf{(-5.34 at 1 percent, -4.80 at 5 percent and -4.58 at 10 percent)}.\par
Finally, the test was carried out on the Real Exchange Rate series where the results \textbf{(-4.003)} show that the test does not reject H0 with the change in the intercept and trend \textbf{(-5.57 at 1 percent, -5.08 at 5 percent and -4.82 at 10 percent)}, in the same way \textbf{(-3.672)} H0 is not rejected in its way of minimizing lags \textbf{(-5.34 at 1 percent, -4.80 at 5 percent and -4.58 at 10 percent)}.\par
In conclusion, it can be seen that the previous series in their seasonally adjusted form are non-stationary because they do not reject H0 in most cases. So we proceed to analyze the growth of each variable by differentiating them. The series in first differences are shown below:\par
Similarly, in each of the series, the Unit Root test was applied using the Dickey-Fuller and Zivot and Andrews method, however, the trend is not taken into account since it is not observed in the previous graphs, the hypotheses are the following:\par
\begin{center}
\textbf{H0} = Has unit root.\par
\textbf{HA} = The series is stationary.
\end{center}
In the first place, the series of Remittances in first differences shows in the results \textbf{(-11.851 in the DF test and -5.274 in the ZA test)} that reject H0 in both tests \textbf{(-3.518 at 1 percent, -2.895 at 5 percent and - 2.582 at 10 percent in DF and -5.34 percent at 1 percent, -4.80 at 5 percent and -4.58 at 10 percent in ZA)} affirming that said integrated series of order I is stationary.\par
Secondly, the series of the Gross Domestic Product of Mexico in first differences shows in the results \textbf{(-23.585 in the DF test and -4.698 in the ZA test)} that rejects H0 in both tests \textbf{(-3.518 at 1 percent, -2.895 at 5 percent and -2.582 at 10 percent in DF and -5.34 percent at 1 percent, -4.80 at 5 percent and -4.58 at 10 percent in ZA)} affirming that said integrated series of order I is stationary.\par
Third, the series of the Gross Domestic Product of the United States in first differences shows in the results \textbf{(-5.997 in the DF test and -6,914 in the ZA test)} that reject H0 in both tests \textbf{(-3.518 at 1 percent, -2.895 at 5 percent and -2.582 at 10 percent in DF and -5.34 percent at 1 percent, -4.80 at 5 percent and -4.58 at 10 percent in ZA)} affirming that said integrated series of order I is stationary.\par
Finally, the series of the Real Exchange Rate in first differences shows in the results \textbf{(-8.578 in the DF test and -9.5 in the ZA test)} that reject H0 in both tests \textbf{(-3.518 at 1 percent, -2.895 at 5 percent and -2.582 at 10 percent in DF and -5.34 percent at 1 percent, -4.80 at 5 percent and -4.58 at 10 percent in ZA)} affirming that said integrated series of order I is stationary.\par
In conclusion, in all integrated series of order I H0 is rejected and, therefore, they are stationary; so, a cointegration analysis will be carried out. In the first place, the optimal lags will be determined using a VAR model; in turn, the dichotomous variables that were mentioned in previous sections as exogenous variables are added.\par
The results show that according to the Bayesian Information Criterion the order one \textbf{(-16.5164)} is determined, the Hannan-Quinn Criterion stipulates an order of four \textbf{(-17.9533)} and the Akaike Information Criterion determines the order twelve \textbf{(-19.4033)}; so we proceed to work with both equations.\par
For the first equation with one lag, the number of cointegration equations is calculated using the \textit{“Johansen method”} (Morán, Bucybaruta, Rivera, 2013). According to the Information Criteria \textbf{(-17.11302 in SBIC and -17.41945 in HQIC)} it can be noted that there are three cointegration equations that will help us to create the Error Correction Model (VEC) by adding the lags of the equation that we are elaborating.\par
In the second equation with four lags, the cointegration equations were calculated where the Information Criteria \textbf{(-17.0929 in SBIC and -18.19574 in HQIC)} indicate that there are three cointegration equations that will help us to develop the Error Correction Model.\par
In the last equation with twelve lags, the cointegration equations were calculated where the Information Criteria \textbf{(-16.74103 in HQIC)} indicate that there are three cointegration equations that will help us to develop the Error Correction Model.
After analyzing the previous results, we proceed to develop the impulse response analysis which will help us to know the growth in the short and long term according to the dependent variable and the independent variables. \par
According to the results obtained, the impact of remittances on the economy that receives them is interesting, however, the behavior of the variables of the country that issued the remittances is strengthened or weakened depending on the currency of said country, the results they will be explained in the next section.\par
\end{frame}