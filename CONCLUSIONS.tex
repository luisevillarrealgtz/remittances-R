\section{Conclusions.}
    \begin{frame}{Conclusions.}
Remittances that enter Mexico have positioned themselves as a benchmark for the country's economic growth over the years, since migrants living in the United States have committed to supporting their families at all costs and supporting their quality of lifetime. With the passing of the years we can observe that records of remittances are broken from the northern country despite the economic complications that may occur.\par
What is important in this research and with the support of the review of the literature presented above is the impact of the variables that depend on the sending and receiving economy of remittances. In the first place, analyzing the economic growth of the United States, it is clear that in the short term any increase in the United States economy positively affects the growth in remittances. This conclusion is based on the fact that there are more jobs and opportunities that are not available in Mexico.\par
The other side of the coin shows us that any growth that the Mexican economy has has a faster impact on whether or not the decision to emigrate to the neighboring country is taken, since many people who live in the United States decide to stay working their entire lives. in said country and affirming that the economic conditions are different from those of Mexico.\par
In addition to this, the exchange rate between the Mexican peso and the US dollar behaves in a similar way to the Mexican economy, and it is logical to think that if the dollar strengthens the decisions to continue working in the United States will be even more solid since it generates greater income for the people who receive it.\par
In the long term, we can see that any positive change in the Mexican economy is caused by remittances, and that the exchange rate plays an important role in making decisions about emigrating to the United States in search of better wages and working conditions. \par Finally, it is worth mentioning that you have to work on the salary and employment policy that is being applied by the TMEC, since if they offer salaries and shifts similar to those of the neighboring country, the burden of migrants contributed by Mexico can be reduced.\par
\end{frame}